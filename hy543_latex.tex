% TEMPLATE for Usenix papers, specifically to meet requirements of
%  USENIX '05
% originally a template for producing IEEE-format articles using LaTeX.
%   written by Matthew Ward, CS Department, Worcester Polytechnic Institute.
% adapted by David Beazley for his excellent SWIG paper in Proceedings,
%   Tcl 96
% turned into a smartass generic template by De Clarke, with thanks to
%   both the above pioneers
% use at your own risk.  Complaints to /dev/null.
% make it two column with no page numbering, default is 10 point

% Munged by Fred Douglis <douglis@research.att.com> 10/97 to separate
% the .sty file from the LaTeX source template, so that people can
% more easily include the .sty file into an existing document.  Also
% changed to more closely follow the style guidelines as represented
% by the Word sample file. 

% Note that since 2010, USENIX does not require endnotes. If you want
% foot of page notes, don't include the endnotes package in the 
% usepackage command, below.

% This version uses the latex2e styles, not the very ancient 2.09 stuff.
\documentclass[letterpaper,twocolumn,10pt]{article}
\usepackage{usenix,epsfig,endnotes,enumitem}
\begin{document}

%don't want date printed
\date{}

%make title bold and 14 pt font (Latex default is non-bold, 16 pt)
\title{\Large \bf A Movie Night with Machine Learning}

%for single author (just remove % characters)
\author{
{\rm Michail G.\ Pachilakis}\\
csd3077@csd.uoc.gr
\and
{\rm Iordanis P. Xanthopoulos}\\
csd3161@csd.uoc.gr
% copy the following lines to add more authors
% \and
% {\rm Name}\\
%Name Institution
} % end author

\maketitle

% Use the following at camera-ready time to suppress page numbers.
% Comment it out when you first submit the paper for review.
\thispagestyle{empty}


\subsection*{Abstract}
As the number of movies released continuously grows, viewers are flooded with information about several movie productions, making the simplest questions like "What movie to see tonight?" really hard to answer. Also movie production studios would like to know if a movie could be a commercial success in order to invest money in it.\par In this work \textbf{i.} we provide a simple question interface so the users can find movies matching on his/her creteria, \textbf{ii.} we provide an interface to answer statistic related questions about any movie dataset and \textbf{iii.} we provide predictions about a movies imdb score based on Machine Learning.\\
\\
\textbf{Keywords} Machine Learning; imdb; prediction; spark; statistics; recommendations


\section{Introduction}

A paragraph of text goes here.  Lots of text.  Plenty of interesting
.test.text. \\ 

More fascinating text. Features\endnote{Remember to use endnotes, not footnotes!} galore, plethora of promises.\\

\section{Dataset}

Some embedded literal typset code might 
look like the following :

{\tt \small
\begin{verbatim}
int wrap_fact(ClientData clientData,
              Tcl_Interp *interp,
              int argc, char *argv[]) {
    int result;
    int arg0;
    if (argc != 2) {
        interp->result = "wrong # args";
        return TCL_ERROR;
    }
    arg0 = atoi(argv[1]);
    result = fact(arg0);
    sprintf(interp->result,"%d",result);
    return TCL_OK;
}
\end{verbatim}
}

Now we're going to cite somebody.  Watch for the cite tag.
Here it comes~\cite{Chaum1981,Diffie1976}.  The tilde character (\~{})
in the source means a non-breaking space.  This way, your reference will
always be attached to the word that preceded it, instead of going to the
next line.

\section{Implementation}
We select to implement three different functionalities. The first one targets mostly the movie production studios and is about movie score predictions. The other two are statistics about the movies contained in the dataset and movie recommendations and they mostly target the simple viewers. \par 
For our implementation we used apache spark and the scala programming language. In the subsection 3.1 is described how we implement the movie predictions and in the subsetion 3.2 are described the implementation of movies statistics and the recommendations.
\subsection{Machine Learning}
We implementated the movie score prediction using Machine Learning for this reason we select to use apache spark and scala. Spark provided us with MLlib a scalable machine learning library consisting common learning algorithms and utilities. Also it provides us with spark.ml which aims to provide a uniform set of high-level APIs that help users to create and tune a practical machine learning pipelines. \par 

Firstly we created a case class containg 28 fields, as many as the dataset features, and after parsing the initial dataset we set every feature to it's coresponding class field. We removed from the dataset movies containing commas in their titles because during splitting they generated more features than they should and they generated errors on our comma seperated csv file.\par 

From our initial 28 features we select to keep only those who had meaning for our predictions, so features like "Title", "IMDBLink", "Country" etc. were dropped. We dropped in total 11 features that we thought it would had the least affect in the movie score prediction. \par 

Because our dataset contained a mix of string, double and integer features we had to transform our string features to numeric values. Fortunately spark.ml provides us with the StringIndexer function. StringIndexer converts String values into categorical indices which could me used by machine learning algorithms in ml library.\par 

From our remaing 17 features we had to select only those that would positevly affect prediction of the movie imdb score. To achieve this we correlated our remaining features and excluded those which negatively affected the "imdb\_score" feature. Fortunately again MLlib provided us with the Statistics package which contained the necessary functions for the feature correlation.\par 


\subsection{Statistics and Recommendations}

It can get tricky typesetting Tcl and C code in LaTeX because they share
a lot of mystical feelings about certain magic characters.  You
will have to do a lot of escaping to typeset curly braces and percent
signs, for example, like this:
``The {\tt \%module} directive
sets the name of the initialization function.  This is optional, but is
recommended if building a Tcl 7.5 module.
Everything inside the {\tt \%\{, \%\}}
block is copied directly into the output. allowing the inclusion of
header files and additional C code." \\

Sometimes you want to really call attention to a piece of text.  You
can center it in the column like this:
\begin{center}
{\tt \_1008e614\_Vector\_p}
\end{center}
and people will really notice it.\\

\noindent
The noindent at the start of this paragraph makes it clear that it's
a continuation of the preceding text, not a new para in its own right.


Now this is an ingenious way to get a forced space.
{\tt Real~$*$} and {\tt double~$*$} are equivalent. 

Now here is another way to call attention to a line of code, but instead
of centering it, we noindent and bold it.\\

\noindent
{\bf \tt size\_t : fread ptr size nobj stream } \\

And here we have made an indented para like a definition tag (dt)
in HTML.  You don't need a surrounding list macro pair.
\begin{itemize}
\item[]  {\tt fread} reads from {\tt stream} into the array {\tt ptr} at
most {\tt nobj} objects of size {\tt size}.   {\tt fread} returns
the number of objects read. 
\end{itemize}
This concludes the definitions tag.

\subsection{How to Build Your Paper}

You have to run {\tt latex} once to prepare your references for
munging.  Then run {\tt bibtex} to build your bibliography metadata.
Then run {\tt latex} twice to ensure all references have been resolved.
If your source file is called {\tt usenixTemplate.tex} and your {\tt
  bibtex} file is called {\tt usenixTemplate.bib}, here's what you do:
{\tt \small
\begin{verbatim}
latex usenixTemplate
bibtex usenixTemplate
latex usenixTemplate
latex usenixTemplate
\end{verbatim}
}

\section{Results}

\section{Conclusion}

A polite author always includes acknowledgments.  Thank everyone,
especially those who funded the work. 

\section{Availability}

It's great when this section says that MyWonderfulApp is free software, 
available via anonymous FTP from

\begin{center}
{\tt ftp.site.dom/pub/myname/Wonderful}\\
\end{center}

Also, it's even greater when you can write that information is also 
available on the Wonderful homepage at 

\begin{center}
{\tt http://www.site.dom/\~{}myname/SWIG}
\end{center}

Now we get serious and fill in those references.  Remember you will
have to run latex twice on the document in order to resolve those
cite tags you met earlier.  This is where they get resolved.
We've preserved some real ones in addition to the template-speak.
After the bibliography you are DONE.

{\footnotesize \bibliographystyle{acm}
\bibliography{../common/bibliography}}
lalalalala


\theendnotes

\end{document}